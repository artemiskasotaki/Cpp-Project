\documentclass[12pt]{article}
 
\usepackage[utf8]{inputenc}
\usepackage[greek,english]{babel}
\usepackage{alphabeta}


\begin{document}

\noindent Αντικειμενοστραφής Εργασία Μέρος 2ο \hfill Read Me \#\\
Άρτεμις Κασωτάκη 1115201900040
\\ \\ 

Αρχικά θα ξεκινίσουμε με το Person.h και το Person.cpp\\
Τον κώδικα περικλείεται με τις οδηγίες ifndef, define, και endif για να αποφευχθούν τα προβλήματα διπλής εισαγωγής. \\
Η κλάση Person έχει ιδιωτικά μέλη name, age, role, studentID, grade και ένα στατικό μέλος count για την καταμέτρηση των αντικειμένων.\\
Υπάρχουν δύο κατασκευαστές ένας προεπιλεγμένος και ένας με παραμέτρους.\\
Οι getters και setters χρησιμοποιούνται για την πρόσβαση και την τροποποίηση των ιδιωτικών μελών της κλάσης.\\
Η  dummy είναι μια εικονική  που προστίθεται για να καθιστά την κλάση πολυμορφική.\\
Τώρα για τις friend functions.\\
Οι συναρτήσεις εισόδου και εξόδου έχουν δηλωθεί ως φίλες για να έχουν πρόσβαση στα ιδιωτικά μέλη της κλάσης.\\
Η στατική μέθοδος getCount επιστρέφει τον αριθμό των αντικειμένων της κλάσης.\\
Ο καταστροφέας δηλώνεται ως εικονικός και καθορίζεται default.\\

Για την κλάση person ορίζεται η αρχική τιμή του στατικού μέλους count 0.\\
Ο προεπιλεγμένος κατασκευαστής αυξάνει το count κατά ένα.\\
Ο κατασκευαστής με παραμέτρους αρχικοποιεί τα μέλη της κλάσης με τις τιμές που περνούν ως παράμετροι και αυξάνει τον αριθμό των αντικειμένων.\\
Οι μέθοδοι getter και setter: \\
Οι getName, getAge, getRole, getStudentID, και getGrade επιστρέφουν τις αντίστοιχες τιμές των μελών της κλάσης.\\
Οι μέθοδοι setName, setAge, setRole, setStudentID, και setGrade τίθενται για την αλλαγή των τιμών των μελών της κλάσης.\\
Η συνάρτηση εισόδου ζητάει από τον χρήστη να εισάγει τα δεδομένα όνομα, ηλικία, ρόλο, και αν είναι φοιτητής τότε και τα στοιχεία φοιτητή για το αντικείμενο.\\
Η συνάρτηση εξόδου εκτυπώνει τα δεδομένα του αντικειμένου.\\
Η στατική μέθοδος getCount επιστρέφει τον συνολικό αριθμό των αντικειμένων της κλάσης Person.\\

Στη συνέχεια το Secretary.h και το Secretary.cpp\\
Η κλάση περιέχει δύο ιδιωτικά μέλη ένα vector από δείκτες προς αντικείμενα τύπου Person για τα departmentMembers και ένα vector από δείκτες προς αντικείμενα τύπου Course για τα courses.\\
Ορίζόνται ο προεπιλεγμένος κατασκευαστής και ο καταστροφέας.\\
Μέθοδος findPerson ελέγχει αν υπάρχει ένα άτομο με το όνομα που δίνεται ως όρισμα στο departmentMembers.
Ο κατασκευαστής αντιγραφής και ο τελεστής εκχώρησης υπερφορτώνονται για να αντιγράφουν τα δεδομένα της κλάσης.
Μέθοδος assignProfessorToCourse Αναθέτει έναν Professor σε ένα Course για ένα εξάμηνο.\\
Μέθοδος getDepartmentMembers επιστρέφει το vector departmentMembers.\\
Υπάρχουν επιπλέον μέθοδοι η addCourse για την προσθήκη ενός μαθήματος, η updateCourse για την ενημέρωση ενός μαθήματος και η saveStudentsToFile για την αποθήκευση των φοιτητών σε ένα αρχείο.\\

Για την κλάση Secretary
Ο καταστροφέας ελέγχει και διαγράφει τα αντικείμενα του vector departmentMembers.\\
Προσθέτει ένα αντικείμενο τύπου Person στον vector departmentMembers.\\
Εκτυπώνει τα στοιχεία όλων των αντικειμένων που βρίσκονται στον vector departmentMembers.\\
Διαβάζει τα στοιχεία ενός νέου αντικειμένου τύπου Person από την είσοδο και το προσθέτει στον vector departmentMembers.\\
Η μέθοδος findPerson ελέγχει αν ένα άτομο με συγκεκριμένο όνομα υπάρχει στον vector departmentMembers.\\
Κατασκευαστής αντιγραφής που αντιγράφει τα αντικείμενα του vector departmentMembers.\\
Υλοποίηση του τελεστή εκχώρησης που εκχωρεί νέα αντικείμενα του vector departmentMembers.\\
Η μέθοδος assignProfessorToCourse αναθέτει έναν καθηγητή σε ένα μάθημα για ένα εξάμηνο. Εκτελεί αναζήτηση στα μέλη του τμήματος για να βρει τον καθηγητή που θέλουμε\\
Η μέθοδος getDepartmentMember επιστρέφει τον vector departmentMembers.\\
Μέθοδοι addCourse και updateCourse προσθέτουν ή ενημερώνουν ένα μάθημα στο vector courses.\\
Η μέθοδος saveStudentsToFile αποθηκεύει τα δεδομένα φοιτητών σε ένα αρχείο.\\

Στη συνέχεια το Professor.h και το Professor.cpp\\
Η κλάση περιέχει ιδιωτικά μέλη όπως το department, τα courses που διδάσκει ο καθηγητής, και έναν χάρτη courseSemesters που αντιστοιχεί τα μαθήματα με τα αντίστοιχα στα εξάμηνα.\\
Υπάρχουν ο προεπιλεγμένος κατασκευαστής που δεν αρχικοποιεί κανένα μέλος και ένας προσαρμοσμένος κατασκευαστής που αρχικοποιεί τα μέλη της κλάσης.\\
Ο καταστροφέας για τη διαγραφή των δεικτών που αναφέρονται στα αντικείμενα Course που ανατίθενται στον καθηγητή.\\
Οι μέθοδοι Accessors και Mutators για την ανάγνωση getDepartment, getCourses  και την ενημέρωση των ιδιωτικών μελών της κλάσης.\\
Η μέθοδος assignCourseForSemester ορίζει το ακαδημαϊκό εξάμηνο ενός μαθήματος, αποθηκεύοντας το ζεύγος course και semester στον χάρτη courseSemesters.\\
Στη κλάση Professor ο προεπιλεγμένος κατασκευαστής αρχικοποιεί τα ιδιωτικά μέλη της κλάσης.\\
Ο προσαρμοσμένος κατασκευαστής αρχικοποιεί τα ιδιωτικά μέλη της κλάσης Professor και της κλάσης Person.\\
Ο καταστροφέας δεν πρέπει να αναφέρει τη διαγραφή των αντικειμένων των Course γιατι θα γίνει αυτόματα από τον καταστροφέα της κλάσης Person.\\
Οι μέθοδοι Accessors επιστρέφουν τα ιδιωτικά μέλη της κλάσης.
Η μέθοδος Mutators χρησιμοποιούνται για ενημέρωση των ιδιωτικών μελών της κλάσης.\\
Η μέθοδος assignCourseForSemester ορίζει το ακαδημαϊκό εξάμηνο ενός μαθήματος.\\

Στη συνέχεια το Student.h και το Student.cpp\\
Η κλάση περιέχει ιδιωτικά μέλη το currentSemester, τον professor που επιβλέπει τον φοιτητή, τα enrolledCourses, τις totalCredits, έναν χάρτη courseSemesters και τον grade που έχει πάρει.\\
Υπάρχουν ο προεπιλεγμένος κατασκευαστής που δεν αρχικοποιεί κανένα μέλος και ένας προσαρμοσμένος κατασκευαστής που αρχικοποιεί τα μέλη της κλάσης.
Οι μέθοδοι Accessors και Mutators ορίζονται μέθοδοι για την ανάγνωση getSemester, getProfessor, getEnrolledCourses, getTotalCredits και την ενημέρωση setSemester, setProfessor των ιδιωτικών μελών της κλάσης.\\
Η μέθοδος enrollInCourse προσθέτει ένα μάθημα στη λίστα εγγεγραμμένων μαθημάτων του φοιτητή και αυξάνει τις συνολικές διδακτικές μονάδες.\\
Μέθοδος enterGradeForCourse που καταχωρεί τον βαθμό που έλαβε ο φοιτητής για ένα μάθημα.\\
Η μέθοδος graduate που ελέγχει αν ο φοιτητής έχει ολοκληρώσει επιτυχώς τις σπουδές του και μπορεί να πάρει το πτυχίο.\\
Η μέθοδος enrollInCourseForCurrentSemester οπου εγγράφει τον φοιτητή σε ένα μάθημα για το τρέχον ακαδημαϊκό εξάμηνο.\\
Η μέθοδος modifyStudent επιτρέπει την ενημέρωση των στοιχείων ενός φοιτητή.\\

Στην κλάση Student περιγράφονται ο προεπιλεγμένος και ο προσαρμοσμένος κατασκευαστής.\\
Ο καταστροφέας της κλάσης Student διαχειρίζεται τη διαγραφή των μαθημάτων που έχει εγγραφεί ο φοιτητής.\\
Οι μέθοδοι getSemester, getProfessor, getEnrolledCourses, getTotalCredits επιστρέφουν τα ιδιωτικά μέλη της κλάσης.\\
Οι μέθοδοι setSemester και setProfessor χρησιμοποιούνται για την αλλαγή των τιμών των ιδιωτικών μελών currentSemester και professor.\\
Η μέθοδος enrollInCourse επιχειρεί να εγγράψει τον φοιτητή σε ένα μάθημα. Ελέγχει εάν το μάθημα είναι διαθέσιμο για εγγραφή στο τρέχον ακαδημαϊκό εξάμηνο αν ναι προσθέτει το μάθημα στη λίστα enrolledCourses και αυξάνει τον συνολικό αριθμό των totalCredits.\\
Η μέθοδος enterGradeForCourse καταχωρεί τον βαθμό που έλαβε ο φοιτητής για ένα μάθημα, αφαιρώντας το μάθημα από τη enrolledCourses.\\
Η μέθοδος enrollInCourseForCurrentSemester εγγράφει τον φοιτητή σε ένα μάθημα για το τρέχον ακαδημαϊκό εξάμηνο.\\
Η μέθοδος graduate κάνει τον έλεγχο για την ολοκλήρωση των σπουδών και παραλαβή του πτυχίου.\\

Στη συνέχεια το Course.h και το Course.cpp\\

Έχουμε τα ιδιωτικά μέλη name, semester, teachingUnits, mandatory και professors\\
Υπάρχουν ο προεπιλεγμένος και ο προσαρμοσμένος κατασκευαστής.\\
Ο καταστροφέας διαχειρίζεται την αποδέσμευση της μνήμης που καταλαμβάνεται από τη λίστα των καθηγητών.\\
Οι μέθοδοι getName, getSemester, getTeachingUnits, isMandatory επιστρέφουν τα αντίστοιχα ιδιωτικά μέλη.\\
Οι μέθοδοι setName, setSemester, setTeachingUnits, setMandatory χρησιμοποιούνται για την αλλαγή των τιμών των ιδιωτικών μελών.\\
Η μέθοδος moveCourseToSemester μεταφέρει το μάθημα σε διαφορετικό εξάμηνο.\\
Η μέθοδος isAvailableForEnrollment ελέγχει αν ένα μάθημα είναι διαθέσιμο για εγγραφή από έναν φοιτητή, με βάση το εξάμηνο του φοιτητή.\\
Η μέθοδος inputCourse χρησιμοποιείται για την εισαγωγή δεδομένων για ένα μάθημα.\\
Η μέθοδος addProfessor προσθέτει έναν καθηγητή στη λίστα των καθηγητών του μαθήματος.\\
Η μέθοδος assignProfessor αναθέτει έναν καθηγητή σε ένα συγκεκριμένο εξάμηνο.\\
Τέλος η μέθοδος inputCourse χρησιμοποιείται για την εισαγωγή δεδομένων για ένα μάθημα από το χρήστη, χρησιμοποιώντας τη συνάρτηση getline για την ανάγνωση ολόκληρης της γραμμής εισαγωγής.\\

Τελος στην main.cpp\\
Περιλαμβάνονται διάφορες βιβλιοθήκες όπως η iostream, fstream, sstream και άλλες που χρησιμοποιούνται για εισαγωγή/εξαγωγή δεδομένων από και προς αρχεία.
Εισάγονται επίσης οι κλάσεις Secretary, Student, Professor, και Course που έχουν υλοποιηθεί στα αρχεία Secretary.h, Student.h, Professor.h και Course.h.\\
Η συνάρτηση loadFromFile διαβάζει δεδομένα από ένα αρχείο και δημιουργεί αντικείμενα των κλάσεων Student και Professor, που προστίθενται στο αντικείμενο Secretary.
Η συνάρτηση saveToFile αποθηκεύει τα δεδομένα του αντικειμένου Secretary σε ένα αρχείο.
Υπάρχει μια συνάρτηση displayMenu που εμφανίζει ένα απλό μενού επιλογών.
Η κύρια λούπα του προγράμματος βρίσκεται στη main και χειρίζεται τις επιλογές του χρήστη με ένα switch statement.
Ο χρήστης έχει τη δυνατότητα να προσθέσει ή να τροποποιήσει πληροφορίες για καθηγητές και φοιτητές.
Υπάρχει λειτουργία για την προσθήκη μαθημάτων, με δυνατότητα ανάθεσης καθηγητή σε ένα μάθημα για συγκεκριμένο εξάμηνο.
Υπάρχει λειτουργία για την εγγραφή ενός φοιτητή σε ένα μάθημα.
Υπάρχει λειτουργία που εμφανίζει τα μέλη του τμήματος.
Υπάρχει λειτουργία που αποθηκεύει τα δεδομένα σε ένα αρχείο.
Ο χρήστης επιλέγει ενέργειες μέσω του μενού και εισάγει δεδομένα όπως ονόματα, ηλικίες και άλλες πληροφορίες για τα μέλη του τμήματος.


\end{document}