\documentclass[12pt]{article}

\usepackage[utf8]{inputenc}
\usepackage[greek,english]{babel}
\usepackage{alphabeta}
 
\usepackage{amsmath,amsthm,amssymb,color,latexsym}
\usepackage{geometry}        
\geometry{letterpaper}    
\usepackage{graphicx}


\begin{document}
\noindent Άρτεμις Κασωτάκη 1115201900040 \hfill Εργασία Μέρος Α\#\\
\\
\\
Λίγα λόγια πίσω απο τον κώδικα της εργασίας:
\\ \\
\textbf{Στο αρχείο Person.cpp }
\\
Ξεκινάμε με τον ορισμό της Person στο "Person.h". Με τις ιδιωτικές μεταβλητές name, age, role, studentID, και grade και την count για να μετρά πόσα 
αντικείμενα έχουν δημιουργηθεί. Τις δημόσιες μεθόδους όπου είναι οι constructors για την αρχικοποίηση των αντικειμένων Person, getters και setters για κάθε όνομα, ηλικία, ρόλος κλπ.
Την  getCount για την ανάκτηση του αριθμού των αντικειμένων και 
operator \guillemotright{}  και operator \guillemotleft{} μαζί με το std::istream και std::ostream.
\\
Επιπλέον υλοποιούνται οι constructors όπου δημιουργούν νέα αντικείμενα Person και αυξάνουν τον count.
οι destructors όπου μειώνουν τον count όταν ένα αντικείμενο Person καταστραφεί.
Οι getters και setters όπου χρησιμοποιούνται ανάκτηση και την αλλαγή των ιδιοτήτων.
και οι operator \guillemotright{}  και operator \guillemotleft{} που επιτρέπουν την εκτύπωση και την είσοδο των δεδομένων από και προς τα αντικείμενα της Person.
Τέλος, η  static count για να παρακολουθεί τον αριθμό των αντικειμένων στο Person που έχουν δημιουργηθεί και καταστραφεί.
\\
\\
\textbf{Στο αρχείο Person.h }
\\
Έδω περιέχεται το πλαίσιο της κλάσης Person μαζί με τις
μεθόδους και τα μέλη των δεδομένων της για τη διαχείριση δεδομένων των ατόμων.
\\
Η κλάση Person περιέχει private members όπως το όνομα, την ηλικία, τον ρόλο, το studentID, τον μέσο όρο grade και μια static count 
για την καταγραφή των αντικειμένων Person.
Έχουμε έναν default constructor που δεν δέχεται παραμέτρους και έναν constructor με παραμέτρους για την αρχικοποίηση των μελών της κλάσης.
Ο destructor για τον καθαρισμό μνήμης.
Οι getters για την ανάκτηση των τιμών των μελών και setters για την τροποποίησή τους.
Οι friend functions operator \guillemotleft{} και operator \guillemotright{} για εισόδο και εξόδο από και προς τα αντικείμενα.
Τέλοσ getCount που επιστρέφει τον συνολικό αριθμό των αντικειμένων.\\
\\
\textbf{Στο αρχείο Secretary.cpp }
\\
Ο default constructor Secretary::Secretary() δεν κάνει κάποια συγκεκριμένη λειτουργία.
Ο destructor Secretary::~Secretary() εκτελείται όταν ένα αντικείμενο Secretary καταστραφεται.
Απελευθερώνει τη μνήμη, καλώντας τον destructor για κάθε αντικείμενο Person που έχει αποθηκευτεί εκεί και εκκαθαρίζει το vector.
Το += προσθέτει έναν δείκτη σε ένα αντικείμενο του Person στο τέλος του departmentMembers.
Ο operator\guillemotleft{} επιτρέπει την εκτύπωση των μελών του departmentMembers.
Ο operator \guillemotright{} προσθέτει ένα νέο αντικείμενο Person στο departmentMembers με το std::istream.
Το  findPerson() ελέγχει αν υπάρχει συγκεκριμένο όνομα στη λίστα departmentMembers.
Ο copy constructor Secretary::Secretary(const Secretary \&other) και ο operator = υλοποιούν τη βαθμολογία.
Και τα δύο αντιγράφουν τη λίστα departmentMembers από ένα άλλο αντικείμενο του Secretary. \\
\\
\textbf{Στο αρχείο Secretary.h }
Η κλάση Secretary να διαχειρίζεται μια λίστα αντικειμένων Person με τα μέλη ενός τμήματος και παρέχει λειτουργίες για τη διαχείριση τους.
Η private std::vector<Person *> departmentMembers είναι ένα vector που περιέχει δείκτες προς αντικείμενα του Person. Αποθηκεύει τα μέλη του τμήματος.
Οι public τώρα,

Secretary(), constructor για τη δημιουργία αντικειμένου στο Secretary.
~Secretary(), destructor για τον καθαρισμό στο Secretary.
Η void operator+=(Person *person) προσθέτει ενα αντικειμένο Person στο vector του τμήματος.
Η friend std::ostream \&operator \guillemotleft{} std::ostream \&out, const Secretary \&secretary) εκτυπώνει τις πληροφορίες σε ένα std::ostream.
Η friend std::istream \&operator \guillemotright{} (std::istream \& in, Secretary secretary) που κάνει είσοδο πληροφοριών μιας γραμματέα από ένα std::istream.
Η bool findPerson(const std::string \&name) const επιστρέφει true αν ένα όνομα ενός ατόμου υπάρχει στο τμήμα της γραμματέας αλλιώς false.
Η Secretary(const Secretary \&other) για την αντιγραφή.
Τέλος η Secretary \&operator=(const Secretary \&other) για την υπερφόρτωση του τελεστή εκχώρησης.\\
\\
\textbf{Στο αρχείο main.cpp}
Στην main του προγράμματος δημιουργούνται αντικείμενα  Person και ένα αντικείμενο Secretary. Στη συνέχεια προσθέτει τα αντικείμενα Person στο Secretary και εκτυπώνει τα μέλη του τμήματος, ψάχνει κάποια 
συγκεκριμένα άτομα στο Secretary και τελικά εκτυπώνει τον συνολικό αριθμό των αντικειμένων Person που φτιάχτηκαν.


\end{document}